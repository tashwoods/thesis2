\chapter{Introduction}
In general, humanity has continually strived to understand the structure and dynamics of reality for widely varying reasons. Each academic field uses a specific set of concepts and models to describe nature. Physics is one such field, that uses mathematical objects to systematically develop testable models about the universe. Currently, the most fundamental types particles are fermions and bosons. Fermions are the particles that make up the "ordinary" matter of the universe, while bosons are the quanta of the fundamental forces. The Standard Model (SM) of particle physics describes the quantum behavior of three of the four fundamental forces: electromagnetic, strong, and weak forces. 

The Standard Model has consistently described much of reality to an extreme degree of accuracy. It has predicted cross sections for strong and electroweak processes that span over ten orders of magnitude [see Fig ~\ref{fig:SM cross sections}] and contains no known logical inconsistencies. Despite the reality of the Standard Model, it still fails to describe aspects of reality and suffers from aesthetic issues. 
The SM fails to account for dark matter, dark energy, neutrino masses, the hierarchy of the fundamental force strengths, and other issues that may have not been noticed yet! This incompleteness may indicate that a more fundamental theory exists. It is hoped that such a theory would address the aforementioned phenomena and the ad-hoc structure and parameter values of the SM. In particular the relative scales of the fundamental forces impose oddly fine-tuned SM parameters, unless there is additional structure at higher energies (e.g. between the electroweak and Planck scales). This and other theoretical arguments motivate the search for new physics at the TeV scale. The set of theories that hope to explain more of reality are known as Beyond the Standard Model theories (BSM). Many of these theories, if true, would revolutionize concepts of symmetry and space-time, which would be intrinsically meaningful.

To probe the physics at this high energy frontier, physicists often collide energetic particles that combine to produce massive particles, such as the Higgs boson and top quark. The more energetic the colliding particles are the more massive produced particles can be. Currently, the world's highest energy particle collider is the Large Hadron Collider (LHC) at the European Organization for Nuclear Research (CERN). 

This thesis presents a search for $WW$ and $WZ$ resonances using data from $pp$ collisions at $\sqrt{s}=13$ TeV using the ATLAS detector at CERN, corresponding to an integrated luminosity of 139 fb$^{-1}$. Diboson resonances are predicted in a number of BSM theories, such as Extended Gauge Models and Extra dimensional models. This search looks for resonances where one $W$ boson decays leptonically and the other $W$ or $Z$ boson decays hadronically. This search is sensitive to diboson resonance production via vector-boson fusion as well as quark-antiquark annihilation and gluon-gluon fusion mechanisms (which will be collectively called non-VBF modes).

To search for these new resonances, Monte-Carlo simulations are used to model SM backgrounds and BSM signals. In these simulations, a series of optimized cuts are used create signal regions (SR) to identify the leptonic and hadronic decay products of the resonance, maximize signal acceptance, and minimize background contamination. In these regions, the resonance mass is calculated as the combined system mass of the leptonic and hadronic system. The expected resonance mass distribution from the simulated backgrounds and anticipated signal are compared to data to search for the existence of these BSM signals (also known as a "bump hunt"). Control regions enriched in the dominant backgrounds, $t\bar{t}$ and $W$+jets (TCR and WCR, respectively) are constructed to be orthogonal to SRs and used to determine the normalization of the $t\bar{t}$ and $W$+jets backgrounds in SRs.

The VBF $W'$ and $Z'$ and ggF $W'$ and $Z'$ resonances studied have unique SR and CR selections to maximize analysis sensitivity. RS Graviton signals are probed using the same selections as the ggF $Z'$ signal. Additionally, more massive resonances are more likely to have boosted $W/Z$ bosons. As the boost of the hadronically decaying boson increases the separation of its hadronic decay products decreases. When the hadronically decaying boson has sufficient boost, the two quarks will overlap and not be identified separately. For this reason, a set of "resolved" selections are used when the hadronic decay products are reconstructed separately, and "merged" selections when the decay products overlap and identified as a single object in the event. A $W/Z$ tagger identifies merged jets as originating from a $W/Z$ bosons based on jet substructure and mass cuts. However, the more boosted the jet is the less likely it is to pass the jet substructure cut. Consequently, the merged selection uses a high purity region, which requires that the jet pass both cuts, and low purity region where the jet can fail the jet substructure cut. 

The aforementioned SR definitions veto events with $b$-jets to minimize $t\bar{t}$ contamination. However, $b$-jets are anticipated from $W'$ resonances from the hadronically decaying $Z$ boson. To increase the signal acceptance of these resonances, a $Z\rightarrow bb$ tagger is used to construct additional SR and CRs called the "tagged" regions (and "untagged" if the event fails the $Z\rightarrow bb$ tagger). 

For each signal model, the simulated and measured resonance mass distributions in the relevant SR and CRs are combined to construct a likelihood. This likelihood is parameterized by the signal strength parameter, $\mu$ and systematic uncertainties of the resonance mass distribution. This likelihood is used to quantify the likelihood of a certain signal model given the anticipated backgrounds and measured data.

No significant excess of events is observed with respect to the Standard Model backgrounds, and constraints on the masses of new W', Z', and bulk-RS Gravitons are extended to up to 3.3 TeV, depending on the model. As the dominant backgrounds in this search contain gluons, classifying jets as quark-initiated or gluon-initiated would improve the sensitivity of this analysis to new physics. Towards this end, this thesis provides a calibrated quark-gluon tagger based on the multiplicity of charged particles within a jet.