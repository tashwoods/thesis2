In general, humanity has continually strived to understand the structure and dynamics of reality for widely varying reasons. Each academic field uses a specific set of concepts and models to describe nature. Physics is one such field, that uses mathematical objects to systematically develop testable models about the universe. Currently, the most fundamental types particles are fermions and bosons. Fermions are the particles that make up the "ordinary" matter of the universe, while bosons are the quanta of the fundamental forces. The Standard Model (SM) of particle physics describes the quantum behavior of three of the four fundamental forces: electromagnetic, strong, and weak forces. 

The Standard Model has consistently described much of reality to an extreme degree of accuracy. It has predicted cross sections for strong and electroweak processes that span over ten orders of magnitude [see Fig ~\ref{fig:SM cross sections}] and contains no known logical inconsistencies. Despite the strength and reality of the Standard Model, it still fails to describe aspects of reality and suffers from aesthetic issues. 
The SM fails to account for dark matter, dark energy, neutrino masses, the hierarchy of the strengths of the fundamental forces, and other issues that may have not been noticed yet! This incompleteness of the SM may indicate that a more fundamental theory exists. It is hoped that such a theory would address the aforementioned phenomena and explain and motivate the ad-hoc structure and parameter values of the SM. One of the fundamental limitations of the SM is that it does not describe gravity. In the lower energy regime (e.g. electroweak scale), the strength of gravity is negligible in comparison to the other forces. In the higher energy regimes, near the Planck scale, the strength of gravity is non-negligible in comparison to the other forces 

This thesis presents a search for $WW$ and $WZ$ resonances using data from $pp$ collisions at $\sqrt{s}=13$ TeV using the ATLAS detector, corresponding to an integrated luminosity of 139 fb$^{-1}$. Diboson resonances are predicted in a number of Standard Model (SM) extensions, such as Extended Gauge Models, and Extra dimensional models. This search looks for resonances where one $W$ boson decays leptonically and the other $W$ or $Z$ boson decays hadronically. This search is sensitive to diboson resonance production via vector-boson fusion as well as quark-antiquark annihilation and gluon-gluon fusion mechanisms. 


No significant excess of events is observed with respect to the Standard Model backgrounds, and constraints on the masses of new W', Z', and bulk-RS Gravitons are extended to up to 3.3 TeV, depending on the model. As the dominant backgrounds in this search contain gluons, classifying jets as quark-initiated or gluon-initiated would make this analysis more sensitive to new physics. Towards this end, this thesis provides a calibrated quark-gluon tagger based on the multiplicity of charged particles within a jet.